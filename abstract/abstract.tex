\documentclass[a4paper]{article}
%\usepackage{simplemargins}

%\usepackage[square]{natbib}
\usepackage{amsmath}
\usepackage{amsfonts}
\usepackage{amssymb}
\usepackage{graphicx}
\usepackage{hyperref}

\begin{document}
\pagenumbering{gobble}

\Large
 \begin{center}
Application of Probabilistic Graphical Models in Forecasting Crude Oil Price\\ 

\hspace{10pt}

% Author names and affiliations
\large
Danish A. Alvi \\

\hspace{10pt}

\large
Supervisor: Dr. Philip Treleaven, Denis de Montigny \\

\hspace{10pt}

\today

\hspace{10pt}

\end{center}

\hspace{10pt}

\normalsize
We propose using Probabilistic Graphical Models such as Belief Networks and Hidden Markov Models (HMMs) to construct a global-macro trading strategy of the Crude Oil Markets. The project involves retrieving data from a number of different sources such the \href{https://www.eia.gov/}{Energy Information Administration} and the \href{https://fred.stlouisfed.org/}{Federal Reserve Economic Data, St. Louis}. We use \href{https://github.com/lopatovsky/HMMs}{\texttt{hmms}}, a Python library implementing a number of different algorithms for HMMs such as the \texttt{Viterbi Algorithm} and the \texttt{Baum-Welch Algorithm}. We have used \href{https://github.com/pgmpy/pgmpy}{\texttt{pgmpy}}, a Python library for Probabilistic Graphical Models for implementing, learning, and carrying out inferences on Belief Networks. \\

We had weekly meetings with our supervisor Dr. Philip Treleaven where we presented our progress with the project. We had presented our background and literature review in January and were completing the implementation of the project during the remaining time. I also regularly reported to my PhD supervisor, Denis De Montigny on Slack, who assisted me in the design process of the project and analysed the projects implementation stage very exhaustively. We managed to complete the implementation of the project by early March after which the rest of documentation of the thesis was completed and finalised.\\

We have implementated of a Proof-of-Concept of a global-macro oil trading strategy using graphical models. The proposed strategy is highly dynamic, allowing oil traders to provide a fundamental expert structure of the oil markets, from which the model will learn the remaining structure of the oil markets. The proposed model performs better than existing forecasts of the oil markets such as the Energy Information Adminstration's Short-term Energy Outlook, however, does not always perform better than actual market direction.

\end{document}
