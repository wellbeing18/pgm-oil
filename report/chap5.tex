\chapter{Conclusions \& Future Work}

The contribution of this thesis to science is to address scientific challenges of conducting forecasting the price of oil in real world operational settings. The proposed Computational Finance and Machine Learning methodologies are highly applicable in a number of real world settings. Firstly, they analyse the structure of the oil markets by constructing a graphical model associating different macroeconomic and physical market factors \textit{without} any expert assistance, thus enabling speculators, risk managers, and energy policy makers to have a greater understanding of the structure of the oil markets. Secondly, it analyses the risk in the energy markets by providing forecasts using current economic situation as evidence. Thirdly, it provides an automated trading mechanism which learns and improves its trading decisions as time passes, utlimately resulting in a higher alpha for a commodities trader. \\

In conclusion, this dissertation contributes to existing literature in a number of ways. Firstly, it contributes to the original research of replacing EGARCH-M derived views with Bayesian Model derived views for the Black-Litterman model \cite{Beach2007}. Secondly, the idea of using time-series data discretised by Hidden Markov Models as inputs to Belief Networks is novel. Thirdly, the dissertation presents a working trading mechanism which is directly deployable in the commodity markets and is entirely capable of independent decision making. Secondly, given the structural learning techniques employed, it is almost an autonomous decision making system which requires absolutely no prior expert knowledge, other than the selection of datasets which is carried out by commodity market analysts. Fourthly, the idea of using time-series data discretised by Hidden Markov Models as inputs to Belief Networks is novel. Fifthly, the level of abstraction for graphical models provided by the Python modules allowed us to understand the design process in theoretical context.  Sixthly, it provides forecasts using a systematic, event-driven,  global macro strategy which takes in account mega geopolitical and macroeconomic changes, thus is capable of generating a higher return than funds based on a high-frequency or fixed-income strategy. Seventhly, it allows us to construct better models of energy markets, hence allowing energy policy makers to understand the underlying structure of the oil markets and use it to respond to different economic events by drafting more effective and sound policy. Lastly, this research is amalgamating a multitude of existing research and experimentation in multiple disciplines and applying it in the commodity markets, in an effort of increasing the alpha for quantitative commodity traders. \\


\section{Future Work}

Though this research successfully pursued objectives with predefined scopes, it also inspired future research research direction based on existing work. A number of subsequent topics may worth further investigation as a continuation of this research. Firstly, we could use explore new and more optimized Structure learning algorithms such as max-min hill climbing algorithm which would enhance the structure learning process by scaling up the datasets involving hundreds of variables \cite{brown2004novel}. Secondly, future research could go beyond using existing probabilisitic graphical models. New graphical models could be constructed either entirely on a new foundation or by combining features of existing graphical models. There are a number of new probabilistic graphical models have been constructed for algorithmic trading, such as Gated Bayesian Networks \cite{bendtsen2017gated}. Thirdly, we could incorporate concepts from reinforcement learning in order to construct better models of the commodities markets by modelling returns as reward to actions \cite{nevmyvaka2006reinforcement}.  Fourthly, we could apply concepts from Game Theory in order to model relationships between OPEC and non-OPEC producers, in an attempt to understand OPECs behaviour by trying to maintain supply at a certain level to maintain an oil price \cite{opecnonopec}. Finally, we could be using different market strategies such as arbitrage, hedging, and pairs trading in order for shorter-term trading.  